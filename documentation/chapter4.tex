\documentclass{report}

\begin{document}
%%%%%%%%%%%%%%%%%%%                 analysisanddesign
\chapter{System Study, Analysis and Design}

\section{Overview of the System}

The UNIBOND mobile application was conceptualized to enhance alumni engagement for the University of Vavuniya. The system study began with identifying the communication gaps between alumni and the university, as well as the absence of a centralized platform for networking, event updates, and academic news.

Stakeholders, including students, alumni, and faculty, were consulted to gather requirements. Key insights highlighted the need for a secure login system, personalized user profiles, job postings, event announcements, and the ability for alumni to connect and share updates.

\section{System Analysis}

Following the requirement gathering phase, the analysis focused on understanding how the proposed mobile application could meet user expectations. The core modules identified were:

\begin{itemize}
    \item \textbf{Authentication System} – Validates users based on their university credentials to restrict access to legitimate alumni.
    \item \textbf{User Profile Module} – Displays user-specific data such as graduation year, department, and contact information.
    \item \textbf{Job Posting and Listings} – Allows institutions and alumni to share career opportunities.
    \item \textbf{Event Announcements} – Notifies users of university events and alumni meetups.
    \item \textbf{Communication Features} – Enables messaging or posting updates for community interaction.
\end{itemize}

System modeling tools like use case diagrams and data flow diagrams (DFDs) were considered to visualize system functionality and data interactions. The backend was powered by Supabase, selected for its real-time data capabilities and ease of integration with React Native.

\section{System Design}

The system design phase focused on developing a scalable and maintainable architecture. The frontend was built using React Native with the Expo framework to ensure cross-platform compatibility. TypeScript was used for type safety and improved code maintainability.

A modular design approach was adopted to separate concerns and improve readability. Navigation was handled using React Navigation and React Native Paper was used for UI components to maintain a consistent and modern user interface.

The database schema on Supabase was structured with separate tables for users, job posts, events, and profiles. Security rules were implemented to ensure authorized access to data based on user roles.

Wireframes and UI mockups were created during the design phase to finalize the app layout and ensure usability. Feedback from early testing was incorporated into the final design before development began.

\newpage
%%%%%%%%%%%%%%%%%%%%%%%%%%%%%%%%%%%%%%%%%%%%%%%%%%%%%%%%%%%%%%%%%%%%%%



\end{document}
